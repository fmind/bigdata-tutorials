% document
\documentclass[a4paper]{article}

% packages
\usepackage[box]{automultiplechoice}
\usepackage[utf8x]{inputenc}
\usepackage[T1]{fontenc}

% parameters
\def\multiSymbole{$\spadesuit$}

\begin{document}

% default scoring
\scoringDefaultS{b=1,v=0,m=-0.25,e=-0.25}
\scoringDefaultM{v=0,e=-0.25,p=-0.5,formula=NBC*(1/NB)-NMC*(1/NB)}

% number of variations
\onecopy{5}{

\vspace*{.5cm}

% title and student name
\begin{minipage}{.4\linewidth}
  \centering\large\bf Big Data\\Validation of Learning\\University of Luxembourg
\end{minipage}
\namefield{\fbox{
\begin{minipage}{.5\linewidth}
  \centering Firstname / Lastname:

  \vspace*{.5cm}\dotfill
  \vspace*{1mm}
\end{minipage}
}}

% description
\begin{center}\em
\bf Duration: 90 minutes.

\vspace*{.5cm}

No documents allowed. The use of electronic calculators or computer is forbidden.

Questions with the \multiSymbole{} sign may have one or several correct answers.

Negative points may be attributed to incorrect answers!
\end{center}

\vspace{1ex}

\section{Relational Databases - Reminders}

\begin{question}{1.1}
  What is a database?
  \begin{choices}
    \wrongchoice{A software that manages more than 1 GB of data}
    \wrongchoice{A software that returns data to a web browser}
    \correctchoice{A software that organizes collection of data}
    \wrongchoice{A software that supports SQL queries}
  \end{choices}
  \explain{example: PostgreSQL, SQLite, MongoDB, Neo4j \ldots}
\end{question}

\begin{questionmult}{1.2}
  Which schema constraint can be used in a relational database?
  \begin{choices}
    \correctchoice{a type (e.g., integer, string, binary …)}
    \correctchoice{a domain (e.g. 0 < age < 150, dept in {1, 2, 3, 4, 5})}
    \wrongchoice{file location (e.g., a relation must be stored in a specific file)}
    \correctchoice{referential integrity (e.g., a foreign key must reference an existing tuple)}
  \end{choices}
  \explain{The goal of a schema is to abstract data definition from physical implementation. Thus, you cannot specify file location.}
\end{questionmult}

\begin{questionmult}{1.3}
  What is the goal of database normalization?
  \begin{choices}
    \wrongchoice{improve query performance by creating indexes that speed up read queries}
    \correctchoice{improve data integrity by forcing a tuple to be only determined by its primary key}
    \correctchoice{reduce data redundancy by preventing the same information to be expressed in multiple rows}
    \correctchoice{avoid data anomalies (e.g., when a course is deleted, the lecture must also be deleted)}
  \end{choices}
  \explain{The goal is to improve database logical structure at the expense of the access performance.}
\end{questionmult}

\begin{question}{1.4}
  Which SQLite operator is used to remove duplicate values?
  \begin{choices}
    \wrongchoice{INTERSECT}
    \wrongchoice{LIMIT}
    \correctchoice{DISTINCT}
    \wrongchoice{EXCEPT}
  \end{choices}
  \explain{example: SELECT DISTINCT firstname from students}
\end{question}

\begin{question}{1.5}
  What is the minimum number of SQLite operators required to retrieve a sorted list of at most 10 tuples from a single relation?
  \begin{choices}
    \wrongchoice{2}
    \wrongchoice{3}
    \correctchoice{4}
    \wrongchoice{5}
  \end{choices}
  \explain{SELECT, FROM, ORDER BY, LIMIT}
\end{question}

\section{Relational Databases - Internals}

\begin{questionmult}{2.1}
  When is it better to store records by attributes (DSM) instead of tuples (NSM)?
  \begin{choices}
    \wrongchoice{to support records with variable size}
    \wrongchoice{for write intensive application (insert)}
    \wrongchoice{to retrieve a record by its primary key (id)}
    \correctchoice{to perform data aggregation (e.g., min, avg)}
  \end{choices}
  \explain{DSM allows you to quickly read a single column and perform aggregation (e.g., average of student grades). On the other hand, attributes are written in separate files, which is not a good fit to retrieve a single record and for write intensive application.}
\end{questionmult}

\begin{question}{2.2}
  What is the worst case complexity of retrieving a single tuple in a relation of size N when there is no index?
  \begin{choices}
    \wrongchoice{O(1) (you can access the tuple in an instant)}
    \correctchoice{O(N) (you have to read the whole relation at most once)}
    \wrongchoice{O(log N) (you can divide the search space by a factor of N)}
    \wrongchoice{O($N^2$) (you have to read the whole relation multiple times)}
  \end{choices}
  \explain{You simply loop through the relation and find the tuple with at most N iterations.}
\end{question}

\begin{question}{2.3}
  Which sentence defines the Consistency property of ACID?
  \begin{choices}
    \correctchoice{any transaction will bring the database from one valid state to another valid state}
    \wrongchoice{once a transaction has been committed, it will remain so, even in the event of a power loss, crashes or error}
    \wrongchoice{the concurrent execution of transaction results in a state that would be obtained if they were executed serially}
    \wrongchoice{if one part of the transaction fails, the entire transaction fails and the database state is left unchanged}
  \end{choices}
  \explain{c.f., course definition}
\end{question}

\begin{questionmult}{2.4}
  Which relational algebra operator can remove tuples from a relation?
  \begin{choices}
    \correctchoice{SELECTION}
    \wrongchoice{PROJECTION}
    \correctchoice{DIFFERENCE}
    \wrongchoice{UNION}
  \end{choices}
  \explain{example: SELECTION(r, age > 20), DIFFERENCE(r1, r2)}
\end{questionmult}

\begin{question}{2.5}
  Given two relations: STUDENT(studentid, name, age) with 30 tuples and GRADE(studentid, class, score) with 20 tuples. What is the number of tuples generated by JOIN(STUDENT, GRADE)?
  \begin{choices}
    \wrongchoice{20}
    \wrongchoice{30}
    \wrongchoice{50}
    \correctchoice{600}
  \end{choices}
  \explain{JOIN performs a Cartesian product: |STUDENT| x |GRADE| = 20 x 30 = 600}
\end{question}

\section{NoSQL databases}

\begin{questionmult}{3.1}
  What are the main benefits of NoSQL databases?
  \begin{choices}
    \correctchoice{improve read/write performance}
    \wrongchoice{support data schema and constraints}
    \wrongchoice{support ACID properties and transactions}
    \correctchoice{partition data across multiple computers}
  \end{choices}
  \explain{The main goal of NoSQL is to improve the read and write scalability of databases through partitioning.}
\end{questionmult}

\begin{question}{3.2}
  Which sentence defines the Consistency property of the CAP theorem?
  \begin{choices}
    \wrongchoice{Consistency is not a property of CAP theorem}
    \correctchoice{All clients have the same view of the data at any time}
    \wrongchoice{The system has to continue working even under arbitrary partitions}
    \wrongchoice{Every request to a non-failed node must result in a correct response}
  \end{choices}
  \explain{c.f., course definition}
\end{question}

\begin{question}{3.3}
  What is the data model of a document database like MongoDB?
  \begin{choices}
    \wrongchoice{single key -> single value}
    \correctchoice{(collection, key) -> value}
    \wrongchoice{(row, column, time) -> value}
    \wrongchoice{a graph G with vertices V and edges E: G = (V, E)}
  \end{choices}
  \explain{c.f., keys are identifiers, values are JSON documents}
\end{question}

\begin{question}{3.4}
  What is the most efficient method to retrieve a single document in MongoDB?
  \begin{choices}
    \correctchoice{use the key of the document}
    \wrongchoice{use the aggregation framework of MongoDB}
    \wrongchoice{perform a find query on the document attributes}
    \wrongchoice{it is not possible to retrieve a single document with MongoDB}
  \end{choices}
  \explain{Documents are always indexed by key, which is the most efficient index.}
\end{question}

\begin{question}{3.5}
  What does this MongoDB query do: “student.find(\{'age': 1\}, \{\})”?
  \begin{choices}
    \wrongchoice{Find the age of all students}
    \correctchoice{Find all students whose age is 1}
    \wrongchoice{Find all students sorted by their age}
    \wrongchoice{Increment the age of all student by 1}
  \end{choices}
  \explain{The first parameter is a SELECTION clause (filter rows). The second parameter is a PROJECTION clause (filter columns).}
\end{question}

\section{MapReduce model}

\begin{questionmult}{4.1}
  What is MapReduce?
  \begin{choices}
    \correctchoice{a system designed for batch processing}
    \wrongchoice{a system designed for streaming processing}
    \correctchoice{a model based on the functional programming paradigm}
    \correctchoice{a framework that support distributed data processing}
  \end{choices}
  \explain{MapReduce is designed to analyzed large amount of data using functional paradigms and distributed processing in batch mode.}
\end{questionmult}

\begin{questionmult}{4.2}
  How can a system based on MapReduce scale?
  \begin{choices}
    \correctchoice{by adding more computer to an existing cluster}
    \wrongchoice{by enforcing ACID properties on the algorithms}
    \wrongchoice{by improving the hardware of existing machines}
    \wrongchoice{by overclocking the computer processing unit}
  \end{choices}
  \explain{The infrastructure scales by adding cheap and convenient machines.}
\end{questionmult}

\begin{question}{4.3}
  Is it possible to build a relational database on top of MapReduce?
  \begin{choices}
    \wrongchoice{no, but the opposite is true}
    \wrongchoice{no, it is not possible at all}
    \correctchoice{yes, using only the Map function}
    \wrongchoice{yes, using only the Reduce function}
  \end{choices}
  \explain{You can implement SELECTION and PROJECTION operations with map.}
\end{question}

\begin{question}{4.4}
  What is the goal of the Reduce function in MapReduce?
  \begin{choices}
    \wrongchoice{to group values in a collection identified by a key}
    \wrongchoice{to apply a function on every element of a collection}
    \wrongchoice{to remove every element when a function output is true}
    \correctchoice{to accumulate a result from a collection starting with an initial element}
  \end{choices}
  \explain{example: reduce can aggregate numbers with an addition or subtraction operator}
\end{question}

\begin{question}{4.5}
  Which map function would you use to count the total number of characters in a collection of words?
  \begin{choices}
    \wrongchoice{map = lambda word: [word, 1]}
    \wrongchoice{map = lambda word: [word, len(word)]}
    \wrongchoice{map = lambda word: ['cnt', 1]}
    \correctchoice{map = lambda word: ['cnt', len(word)]}
  \end{choices}
  \explain{You need to associate the length of each word to a single key that will be used during the reduce stage.}
\end{question}

\section{Hadoop and Spark}

\begin{questionmult}{5.1}
  What are the main limitations of Hadoop compared to Spark?
  \begin{choices}
    \wrongchoice{it cannot process large datasets in parallel}
    \correctchoice{its approach is too minimalist and low-level}
    \correctchoice{it cannot retain intermediate results in main memory (RAM)}
    \correctchoice{it does not provide convenient functions, like filter or flatmap}
  \end{choices}
  \explain{The main improvements Spark provides is a higher-level framework and the ability to store intermediate data on RAM.}
\end{questionmult}

\begin{questionmult}{5.2}
  What are the properties of a Resilient Distributed Dataset (RDD)?
  \begin{choices}
    \wrongchoice{mutable (can be changed)}
    \correctchoice{can be processed in parallel}
    \correctchoice{represent a collection of data}
    \correctchoice{distributed across multiple computers}
  \end{choices}
  \explain{RDDs are collection of data distributed across multiple computers. However, they are not mutable to support parallel operations.}
\end{questionmult}

\begin{questionmult}{5.3}
  Which types of operation can be performed on a RDD?
  \begin{choices}
    \correctchoice{actions}
    \wrongchoice{extraction}
    \wrongchoice{configuration}
    \correctchoice{exploitation}
  \end{choices}
  \explain{Transforms process RDD and create new ones (e.g., filter). Actions perform side-effects (e.g., take).}
\end{questionmult}

\begin{question}{5.4}
  What is the output of map(lambda x: [x, 1]) when the function is applied on the following RDD: [“big data big”]?
  \begin{choices}
    \wrongchoice{[[“big”, 1], [“data”, 1”], [“big”, 1]]}
    \wrongchoice{[[“big”, 2], [“data”, 1]]}
    \correctchoice{[[“big data big”], 1]}
    \wrongchoice{it will raise an error}
  \end{choices}
  \explain{The given RDD contains a single value. Thus, it will simply put it in a new array with the value 1.}
\end{question}

\begin{question}{5.5}
  How many operations are required to compute the number of words that start with letter B in a dataset of sentences?
  \begin{choices}
    \wrongchoice{2}
    \correctchoice{3}
    \wrongchoice{4}
    \wrongchoice{5}
  \end{choices}
  \explain{filter: keeps words with the letter, map: count he number of words, reduce: compute the total with sum }
\end{question}

\section{Datalog systems}

\begin{questionmult}{6.1}
  Which information are required to describe a fact?
  \begin{choices}
    \correctchoice{an entity (e.g. Bob)}
    \correctchoice{an attribute (e.g., loves)}
    \correctchoice{a value (e.g. pizza)}
    \correctchoice{a timestamp (e.g., since today)}
  \end{choices}
  \explain{e.g. Donald Trump (entity) is president of (attribute) the USA (value) since 2017 (timestamp)}
\end{questionmult}

\begin{questionmult}{6.2}
  Which types of transaction are supported by Datalog systems?
  \begin{choices}
    \correctchoice{add a fact}
    \wrongchoice{update a fact}
    \wrongchoice{delete a fact}
    \correctchoice{retract a fact}
  \end{choices}
  \explain{You cannot update or delete facts. You can only add them or retract them if they are not true anymore.}
\end{questionmult}

\begin{question}{6.3}
  What does this Datalog clause do: [42 :email ?email]?
  \begin{choices}
    \wrongchoice{find the list of all emails}
    \wrongchoice{find the list of persons with 42 emails}
    \correctchoice{find the email of a particular person}
    \wrongchoice{find the list of persons with an email attribute}
  \end{choices}
  \explain{Since the entity (42) and the attribute (:email) are fixed, the clause captures the email attribute of entity 42.}
\end{question}

\section{Distributed Streaming}

\begin{questionmult}{7.1}
  Which concepts are related to the Actor Model?
  \begin{choices}
    \wrongchoice{methods}
    \correctchoice{messages}
    \correctchoice{mailboxes}
    \correctchoice{asynchronous}
  \end{choices}
  \explain{Actors communicate by sending asynchronous messages to mailboxes.}
\end{questionmult}

\begin{question}{7.2}
  Which sentence defines the Concurrency execution model?
  \begin{choices}
    \wrongchoice{doing a lot of things at once}
    \wrongchoice{splitting tasks between workers}
    \correctchoice{dealing with a lot of things at once}
    \wrongchoice{dealing with things one after the other}
  \end{choices}
  \explain{The goal of Concurrency is to interleave multiple processes (e.g., download multiple files from the Internet)}
\end{question}

\begin{questionmult}{7.3}
  Which of these components are typical in a distributed streaming system?
  \begin{choices}
    \wrongchoice{NoSQL databases}
    \correctchoice{message brokers}
    \correctchoice{worker processes}
    \wrongchoice{MapReduce infrastructure}
  \end{choices}
  \explain{Brokers store messages that are then distributed across worker processes.}
\end{questionmult}

\section{Statistics and Data Analysis}

\begin{question}{8.1}
  How would you define statistics?
  \begin{choices}
    \wrongchoice{The science of using dataframes}
    \wrongchoice{The science of producing data reports}
    \wrongchoice{The science of creating nice visualization}
    \correctchoice{The science of drawing conclusion from data}
  \end{choices}
  \explain{The goal of statistic is to create meaningful metrics from data.}
\end{question}

\begin{questionmult}{8.2}
  Which of these measures can describe the dispersion of a distribution?
  \begin{choices}
    \wrongchoice{mean}
    \wrongchoice{median}
    \correctchoice{standard deviation}
    \correctchoice{inter-quartile range}
  \end{choices}
  \explain{Mean and median describes the centrality of a distribution. SD and IQR describes the dispersion of a distribution.}
\end{questionmult}

\begin{questionmult}{8.3}
  Which of these measures are not statistically robust?
  \begin{choices}
    \correctchoice{mean}
    \wrongchoice{median}
    \correctchoice{standard deviation}
    \wrongchoice{inter-quartile range}
  \end{choices}
  \explain{Mean and SD are not robust (e.g., when Bill Gates enters in a bar).}
\end{questionmult}

\begin{question}{8.4}
  What is the percentile associated to the 2nd quartile (Q2)?
  \begin{choices}
    \wrongchoice{2}
    \wrongchoice{25}
    \correctchoice{50}
    \wrongchoice{75}
  \end{choices}
  \explain{The 2nd quartile is the 50 percentile (median) that splits a dataset in two sets of equal size.}
\end{question}

\begin{question}{8.5}
  Which method is used to compute the number of items per column pairs?
  \begin{choices}
    \wrongchoice{value\_counts}
    \correctchoice{crosstab}
    \wrongchoice{describe}
    \wrongchoice{corr}
  \end{choices}
  \explain{Crosstab compute the cross table between two variables.}
\end{question}


\section{Communication and Visualization}

\begin{questionmult}{9.1}
  Which of these visual attributes can be used for quantitative variables?
  \begin{choices}
    \correctchoice{position}
    \correctchoice{size}
    \wrongchoice{texture}
    \wrongchoice{shape}
  \end{choices}
  \explain{You can encode a quantitative variable with position (e.g., position of point on an axis) or size (e.g., diameter of a point). You cannot use shape (e.g., triangle, square) or texture (e.g., strip of full) attributes.}
\end{questionmult}

\begin{questionmult}{9.2}
  What are the problems associated with the Curse of Dimensionality?
  \begin{choices}
    \correctchoice{difficult to plot: it is much harder to visualize information}
    \wrongchoice{computer limitation: a dataset can not have more than 3 dimensions}
    \correctchoice{concentration of distances: distances become numerically similar}
    \wrongchoice{difficult to analyze: no technique can reduce the number of dimensions}
  \end{choices}
  \explain{Plots are often limited to 3 dimensions, but not datasets. Distances are harder to interpret. Dimension reductions techniques can be used to address these problems (e.g., PCA).}
\end{questionmult}

\begin{question}{9.3}
  Which plot can be used to present the statistical measures associated to a distribution?
  \begin{choices}
    \wrongchoice{bar plot}
    \correctchoice{boxplot}
    \wrongchoice{histogram}
    \wrongchoice{jitter plot}
  \end{choices}
  \explain{A boxplot displays the median, quartiles, minimum, maximum and possibly the mean of a distribution.}
\end{question}

\section{Supervised Learning}

\begin{questionmult}{10.1}
  Which of these machine learning tasks are associated to Supervised Learning?
  \begin{choices}
    \wrongchoice{clustering}
    \correctchoice{regression}
    \wrongchoice{segmentation}
    \correctchoice{classification}
  \end{choices}
  \explain{Classification: will it be rainy or sunny tomorrow, Regression: what will be the temperature tomorrow}
\end{questionmult}

\begin{question}{10.2}
  What does the K parameter of the K-Nearest Neighbor algorithm refer to?
  \begin{choices}
    \wrongchoice{the number of folds in the dataset}
    \wrongchoice{the number of classes in the dataset}
    \wrongchoice{the number of features in the dataset}
    \correctchoice{something independent of the dataset}
  \end{choices}
  \explain{K is an algorithm parameter which limits the number of items considered by the algorithm. With K=3, it considers 3 neighbors items.}
\end{question}

\begin{questionmult}{10.3}
  Which problems are associated to under fitting?
  \begin{choices}
    \correctchoice{high bias}
    \wrongchoice{high variance}
    \correctchoice{high testing error}
    \correctchoice{high training error}
  \end{choices}
  \explain{Under fitting is associated to high bias. Thus, both training and testing errors will be high.}
\end{questionmult}

\begin{questionmult}{10.4}
  Which features would you keep to predict the survival of Titanic passengers?
  \begin{choices}
    \correctchoice{age}
    \correctchoice{sex}
    \wrongchoice{name}
    \correctchoice{passenger class}
  \end{choices}
  \explain{Passenger name is to unique to provide generic predictions.}
\end{questionmult}

\begin{questionmult}{10.5}
  What can you change to improve the prediction of a machine learning model?
  \begin{choices}
    \correctchoice{the features}
    \correctchoice{the algorithm}
    \wrongchoice{the target output}
    \wrongchoice{the programming language}
  \end{choices}
  \explain{Features and algorithm can be changed to find the best fit. Changing the language would have no effect. Changing the target output would be a fraud.}
\end{questionmult}

\section{Unsupervised Learning}

\begin{question}{11.1}
  What is the concept of Feature Pruning?
  \begin{choices}
    \correctchoice{removing bad features}
    \wrongchoice{selecting good features}
    \wrongchoice{creating random features}
    \wrongchoice{normalizing all features}
  \end{choices}
  \explain{Feature Pruning removes bad features (outlier, random, extreme values \ldots)}
\end{question}

\begin{question}{11.2}
  What is the benefit of normalizing features?
  \begin{choices}
    \wrongchoice{convert to human readable numbers}
    \correctchoice{remove entries that contain outliers}
    \wrongchoice{fill columns that contain missing values}
    \wrongchoice{rescale features so they have the same magnitude}
  \end{choices}
  \explain{Normalization is mandatory for certain machine learning algorithms.}
\end{question}

\begin{question}{11.3}
  What is the difference between supervised and unsupervised learning?
  \begin{choices}
    \wrongchoice{features are not available in unsupervised learning}
    \wrongchoice{algorithms are not available in unsupervised learning}
    \wrongchoice{human knowledge is not available in unsupervised learning}
    \correctchoice{target output are not available in unsupervised learning}
  \end{choices}
  \explain{Contrary to supervised learning, we lack the classes/labels to train the algorithm.}
\end{question}

\begin{questionmult}{11.4}
  Which steps are required by the K-Means clustering algorithm?
  \begin{choices}
    \correctchoice{start with some initial centers}
    \correctchoice{assign each point to the closest center}
    \wrongchoice{find the k nearest neighbors for each point}
    \correctchoice{compute new centers from each point in a cluster}
  \end{choices}
  \explain{K-Means does not find the nearest neighbors like the kNN algorithm.}
\end{questionmult}

\begin{question}{11.5}
  When does the K-Means clustering algorithm converge?
  \begin{choices}
    \wrongchoice{when the prediction score reaches 100}
    \correctchoice{when the cluster centers stay the same}
    \wrongchoice{after the first iteration of the algorithm}
    \wrongchoice{when the distance between cluster centers reach 0}
  \end{choices}
  \explain{K-Means converges when the algorithm does not change the cluster centers.}
\end{question}

}

\end{document}
